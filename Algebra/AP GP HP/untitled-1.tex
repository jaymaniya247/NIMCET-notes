\documentclass[11pt,a4paper,landscape]{article}
\usepackage[utf8]{inputenc}
\usepackage[english]{babel}
\usepackage{multicol}
\usepackage[margin=0.5in]{geometry}
\usepackage{amsmath}
\usepackage{enumerate}
\usepackage{fancyhdr}

\everymath{\displaystyle}
\pagestyle{fancy}
\fancyhf{}
\renewcommand{\headrulewidth}{0pt}
\renewcommand{\footrulewidth}{2pt}

\lfoot{\bfseries Prashant Dhirendra : prashant.dhiru@gmail.com} 
\begin{document}
\begin{multicols}{3}


\begin{tabbing}
{\bfseries {\Large Arithmetic Progression}}\\
{\bfseries General term :} \hspace{10mm} \= $a_n = a+(n-1)d$ \\
{\bfseries $n^{th}$ term from  end  :} \> $a+(m-n)d$ \\
where $n^{th}$ term from  end = $(m-n+1)$ from start\\ \\

{\bfseries if A,B,C are in AP:}
$2B=A+C$\\ \\
$a_1 + a_n = a_2 + a_{n-1}$\\
$2a_n = a_{n+k} + a_{n-k}$\\  \\

{\bfseries selecting middle term of AP:} \\
ODD $\Rightarrow - a +$  \> CD $\rightarrow d$\\
EVEN $\Rightarrow (a-d)(a+d)$  \> CD $\rightarrow 2d$\\ \\

{\bfseries Sum of AP :} \\
with common difference \> $\displaystyle S_n = \frac{n}{2}\{2a+(n-1)d\}$\\ \\ 
with last term \> $\displaystyle S_n = \frac{n}{2}\{a+l\}$
\end{tabbing}

{\bfseries \noindent results of sum of AP :} 
\begin{itemize}
  	\item Seq is AP $\rightarrow$ if sum of n terms form $An^{2}+Bn$
	\item if ratio of sum is given, then ratio of $n^{th}$ term is \\ $\rightarrow$ replace $n$ by $2n-1$
	\item if ratio of $n^{th}$ term is given, then ratio of sum is \\ $\rightarrow$ replace $n$ by $\displaystyle \frac{n+1}{2}$
\\
\end{itemize}

\begin{tabbing}
{\bfseries {\Large Geometric Progression}}\\
{\bfseries General terms  :} \hspace{10mm} \= $a_n = ar^{r-1}$ \\
{\bfseries $n^{th}$ term from  end  :} \> $a_n = ar^{m-n}$ \\
{\bfseries using last term  :} \> $\displaystyle a_n= l \left({\frac{1}{r}} \right)^{n-1}$\\
\end{tabbing}

{\bfseries \noindent Properties of GP :} 
\begin{itemize}
  	\item GP divided/multiplied by constant, stays GP
	\item reciprocal of GP, is GP
	\item $a_1 a_n = a_2 a_{n-1} = a_k a_{n-k+1}$
	\item $b^2 = ac$
	\item for a GP $a_1, a_2 \ldots a_n$\\
		$\Rightarrow GM = \left({a_1 a_2 \ldots a_n} \right)^{\frac{1}{n}}$
	\item if  $a_1,a_2 \ldots a_n \Rightarrow$ GP \\
		$\log a_1, \log a_2 \ldots \log a_n \Rightarrow$ AP and vice-versa
\end{itemize}
{\bfseries \noindent results of sum of GP :} \\\\
$
	S_n =
	\begin{cases}
	a\left(\displaystyle \frac{1-r^{n}}{1-r} \right), & r \neq 1\\
	na, & r
	\end{cases}
$\\ \\ \\
$
	S_n = \displaystyle \frac{a -lr}{1-r}=  \displaystyle \frac{lr-a}{r-1}
$\\ \\ \\
$
	S_\infty =
	\begin{cases}
		\displaystyle \frac{a}{1-r} & \quad -1<r<1
	\end{cases}
$\\
\begin{tabbing}
{\bfseries \noindent {\Large Harmonic Progression}}\\
if $a_1,a_2 \ldots a_n$ are in HP\\ \\
then $\displaystyle \frac{1}{a_1},\frac{1}{a_2} \ldots \frac{1}{a_n}$ are in AP\\ \\ \\
{\bfseries \noindent common difference : }\hspace{2mm} \= $d = \frac{1}{a_2}-\frac{1}{a_1}$\\ \\ \\
{\bfseries \noindent general term : }\> $a_n = \frac{1}{\displaystyle \frac{1}{a_1}+(n-1)d}$
\end{tabbing}
{\bfseries \noindent Harmonic Mean(H)}
\begin{itemize}
	\item $H = \frac{2ab}{a+b}$
	\item $a,H_1,H_2 \ldots,H_n,b$ are in HP\\ \\
		then $\frac{1}{a},\frac{1}{H_1},\frac{1}{H_2} \ldots,\frac{1}{H_n},\frac{1}{b}$ are in AP\\ \\
		$\rightarrow \frac{1}{b} = (n+2)^{th} \\ \\ \rightarrow \frac{1}{b} = \frac{1}{a}+(n+1)D \quad D =\frac{a-b}{(n+1)ab\\ \\}$
\end{itemize}

{\bfseries \noindent {\Large  Arithmetico-geometric Sequence}}\\
$a,(a+d)r,(a+2d)r^2,\ldots$ is a A.G. sequence
\begin{tabbing}
{\bfseries $n^{th}$ term :} \hspace{15mm} \= $a_n = \{a+(n-1)d\}.r^{n-1}$ \\ \\
{\bfseries sum of  $\infty$ term :} \>$S_{\infty} = \frac{a}{1-r}+\frac{d.r}{{(1-r)}^{2}}$ \\
\end{tabbing}

{\bfseries \noindent {\Large Sum of Some Sequence}}
\begin{tabbing}
first $n$ natural no : \hspace{5mm}\=$\frac{n(n+1)}{2}$\\ \\
square of first $n$ no : \> $\frac{n(n+1)(2n+1)}{6}$\\ \\
cube of first $n$ no : \> ${\left \{ \frac{n(n+1)}{2} \right \} }^2$\\\\
$4^{th}$ power of $n$ no : \> $\frac{n(n+1)(2n+1)(3n^{2}+3n-1)}{30}$
\end{tabbing}
\end{multicols}
\pagebreak

%2nd page starts from here
\begin{multicols*}{3}
{\bfseries \noindent {\Large Relation between AM, GM and HM}}\\
\begin{itemize}
	\item {\bfseries \noindent A,G and H between 2 numbers(a and b):}\\
	 	{$$A =\frac{a+b}{2} \qquad G =\sqrt{AB} \qquad H=\frac{2ab}{a+b}$$}

	\item $A>G>H$
	\item quadratic eq having a and b as its roots\\
		$$x^2 - 2Ax+G^2=0$$
	\item the two numbers(a,b) are $A\pm \sqrt{A^2 - G^2}$
	\item if $A$ and $G$ are in the ratio $m:n$,
		then the number(a,b) are in ratio$$m+\sqrt{m^2 - n^2} : m-\sqrt{m^2 - n^2}$$
	\item {\bfseries \noindent A,G and H between 3 numbers(a, b, c):}\\
		$$A=\frac{a+b+c}{3} \quad G=(abc)^{\frac{1}{3}} \quad \frac{1}{H}=\frac{1}{3}\left (\frac{1}{a}+\frac{1}{b}+				\frac{1}{c} \right ) $$
	\item cubic equation where a,b,c are the roots
		$$x^3-3Ax^2+\frac{3G^3}{H}x-G^3=0$$
\end{itemize}
\vfill\null
\columnbreak
{\bfseries {\huge Tips and Tricks}}\\
(black space for tips,tricks and important question) 

\end{multicols*}
\end{document}