\documentclass[11pt,a4paper,landscape]{article}
\usepackage[utf8]{inputenc}
\usepackage[english]{babel}
\usepackage[left=2cm,right=2cm,top=2cm,bottom=2cm]{geometry}
\usepackage{fancyhdr}
\usepackage{multicol}
\usepackage{amssymb}
\usepackage{amsmath}

\author{Himanshu Mittal}
\title{Quadratic Equations}

\renewcommand{\labelenumi}{\theenumi}

\everymath{\displaystyle}
\pagestyle{fancy}
\fancyhf{}

\renewcommand{\headrulewidth}{0pt}
\renewcommand{\footrulewidth}{2pt}
\lfoot{\bfseries Himanshu Mittal : mittal01091997@gmail.com} 

\begin{document}
\begin{multicols*}{3}
{\noindent \bfseries {\LARGE Quadratic Equations}}\\

\section{Introduction}
	\subsection{Definition}
	For Coefficients: $a_0,\,a_1,\,a_2\,\cdots\,a_n$, an equation in the form: $f(n) = a_0+a_1x+a_2x^2+\cdots+a_nx^n$ is called a polynomial of $n$ degree.
	\begin{itemize}
	\item \textbf{Real Polynomial:} Where all coefficients are real numbers.
	\item \textbf{Complex Polynomial:} Where all coefficients are complex numbers.
	\item \textbf{Polynomial Roots:} The Value satisfying the polynomial $f(n) = 0$ is called a root of the Equation.
	\item \textbf{Quadratic Polynomial:} A Equation of degree 2.
	\item \textbf{Cubic Polynomial:} A Equation of degree 3.
	\end{itemize}
	
	\subsection{Position of Roots}
	\renewcommand{\theenumi}{\roman{enumi}}%
	\begin{enumerate}
	\item Every Equation of odd degree must have one real root.
	\item Complex roots always lie in pairs.
	\item Every even degree equation whose last term is negative and coefficient of first term is positive has two real roots, one is +ve and one is -ve.
	\end{enumerate}
	
	\subsection{Discrete Rule of Signs}
	\begin{itemize}
	\item \textbf{Maximum number of +ve real roots:} Number of changes o sign from +ve to -ve and -ve to +ve in $f(x)$.
	\noindent{\textbf{E.g.}}
	\begin{tabular}{cccccc}
	$f(x)=$ & $x^3$ & $+6x^2$ & $+11x$ & $-6$ & $=0$ \\ 
	  & + & + & + & - &   \\ 
	\end{tabular}
	hence equation has at-most one +ve real root.
	\item \textbf{Maximum number of -ve real roots:} Number of changes of sign from +ve to -ve and -ve to +ve in $f(-x)$.
	\noindent{\textbf{E.g.}}
	\begin{tabular}{cccccc}
	$f(x)=$ & $x^3$ & $-6x^2$ & $+11x$ & $+6$ & $=0$ \\ 
	$f(-x)=$ & $-x^3$ & $+6x^2$ & $-11x$ & $-6$ & $=0$ \\ 
	  & - & + & - & - &   \\
	\end{tabular}
	Hence the equation has at-most two -ve real roots.
	\item If all the terms of an Equation are Positive and the equation involves no odd powers of $x$, then all its roots are complex.
	\end{itemize}
	
\section{Relation between roots and Coefficients}
	\subsection{Formation of roots}
	\textbf{Quadratic:} If $\alpha$, $\beta$ are the roots of equation $ax^2+bx+c=0$, then:\\
	\begin{center}
	$\alpha+\beta = \frac{-b}{a}$,\, $\alpha\beta = \frac{c}{a}$
	\end{center}
	
	\textbf{Cubic:} If $\alpha, \beta, \gamma$ are the roots of equation $ax^3+bx^2+cx+d=0$, then:\\
	\begin{center}
	$\alpha+\beta+\gamma = \frac{-b}{a}$\\
	$\alpha\beta+\beta\gamma+\alpha\gamma = \frac{c}{a}$\\
	$\alpha\beta\gamma = \frac{-d}{a}$
	\end{center}
	\subsection{Formation of equation}
	\textbf{Quadratic:} If $\alpha$, $\beta$ are the roots of an equation then equation is:\\
	\begin{center}
	$x^2-(\alpha+\beta)x+\alpha\beta=0$
	\end{center}
	\textbf{Cubic:} If $\alpha$, $\beta, \gamma$ are the roots of an equation then equation is:
	\begin{center}
	$x^3-(\alpha+\beta+\gamma)x^2+(\alpha\beta+\beta\gamma+\alpha\gamma)x-(\alpha\beta\gamma)=0$
	\end{center}
\section{Roots of an Equation}
	\subsection{Roots of quadratic Equation}
	\begin{tabular}{rl}
    For an Equation:& $ax^2+bx+c=0$\\
    Its Discriminant(D) is:& $b^2-4ac$\\
    Its Roots:& $\displaystyle{\frac{-b\pm\sqrt{D}}{2a}}$\\
    or,& $\displaystyle{\frac{-b\pm\sqrt{b^2-4ac}}{2a}}$
	\end{tabular}
	\begin{large}
	\textbf{Nature of Roots}
	\end{large}
	
	\renewcommand{\theenumi}{\roman{enumi}}%
	\begin{enumerate}
	\item $D>0\Rightarrow$ Real and distinct roots.
	\item $D=0\Rightarrow$ Real and equal roots.
	\item $D<0\Rightarrow$ complex roots.
	\item $D$ is a perfect square (and $a,b,c$ is a perfect square)$\Rightarrow$ Roots are rational
	\item $a=1\ \&\ b,c\in \mathbb{Z}$ and then roots are rational $\Rightarrow$ roots are integers
	\item Surd roots of an equation always lie in pairs: \textbf{e.g.} $2+\sqrt{3}\ \&\ 2-\sqrt{3}$\\
	however, if coefficients are irrational, this may not be true.
	\end{enumerate}
	
	\subsection{Common Roots}
	let two quadratic equations be:	$f_1(x)=a_1x^2+b_1x+c_1$ and $f_2(x)=a_2x^2+b_2x+c_2$ and let $\alpha$ be the common root, then:\\
	$\displaystyle{\frac{\alpha^2}{b_1c_2-b_2c_1}=\frac{\alpha}{c_1a_2-a_2c_1}=\frac{1}{a_1b_2-a_2b_1}}$\\
	
	Hence common root $\alpha$ is given by:\\

	$\displaystyle{\alpha=\frac{c_1a_2-a_2c_1}{a_1b_2-a_2b_1}}$ or $\displaystyle{\alpha=\frac{b_1c_2-b_2c_1}{c_1a_2-a_2c_1}}$
	
\section{Miscellaneous}
	\subsection{Transformation of Equation}
	\begin{itemize}
	\item An Equation hose roots are reciprocal of the roots of a given equation is obtained y replacing $x$ by $\displaystyle{\frac{1}{x}}$ in the given equation.
	\item An Equation whose roots are negative of the roots of a given equation is obtained y replacing $x$ by $-x$ in the given equation.
	\item An Equation whose roots are squares of the roots of a given equation is obtained y replacing $x$ by $\sqrt{x}$ in the given equation.
	\item An Equation whose roots are cubes of the roots of a given equation is obtained y replacing $x$ by $\sqrt[3]{x}$ in the given equation.
	\end{itemize}
	\subsection{Maximum and Minimum value of rational expression}
	To find the value attained by a rational expression of the form:
	\begin{center}
	$\frac{a_1x^2+b_1x+c_1}{a_2x^2+b_2x+c_2}$ for real values of $x$,
	\end{center}
	we may use the following algorithm:
	\renewcommand{\theenumi}{\Roman{enumi}}%
	\begin{enumerate}
	\item Equate the given equation to $"y"$
	\item Obtain a quadratic equation in $x$ by simplifying the expression in \textbf{Step I}.
	\item Obtain the discriminant in \textbf{Step II}
	\item Put discriminant from \textbf{Step III} to $\geq 0$ and solve the in-equation for $y$. The values of $y$ so obtained determine the set of values attained by the given rational expression.
	\end{enumerate}
	

%put to do breaaks
%\vfill\null
%\columnbreak
\end{multicols*}
\end{document}
